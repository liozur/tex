%!TEX program = xelatex

% TODO: 让 \endnote 的标记使用黑体数字
% TODO: 让 \endnote 命令可以用于 \title 之中
% TODO: 让 \tdoc 与 \tpart 在目录中可以显示页码范围
% TODO: 修复 \editornote
% TODO: \authornote 与 \editornote 之间的垂直间距可能需要微调

\documentclass[
  zihao=5,
  punct=kaiming,
  linespread=1.4,
  sub4section,
]{ctexbook}

% ------------------------------------------------------------------------------
% 导言区
% ------------------------------------------------------------------------------

% ----------------------------------------------------------
% 宏包加载
% ----------------------------------------------------------

\usepackage{bigfoot}
\usepackage{calc}
\usepackage{CJKfntef}
\usepackage{enotez}
\usepackage{etoolbox}
\usepackage{fancyhdr}
\usepackage{fontspec}
\usepackage{geometry}
\usepackage{graphicx}
\usepackage[hidelinks,hyperfootnotes=true]{hyperref}
\usepackage{lipsum}
\usepackage{perpage}
\usepackage{scrextend}
\usepackage{tikz}
\usepackage{titlesec}
\usepackage{titletoc}
\usepackage{truncate}
\usepackage{ulem}
\usepackage{xcolor}
\usepackage{xunicode-addon}

\raggedbottom
\newsavebox{\headerbox}

\usepackage{xeCJKfntef}
\xeCJKsetup{underdot/symbol={\raisebox{-5pt}{◦}}}
\newcommand{\dotemph}[1]{\CJKunderdot{#1}}

% ----------------------------------------------------------
% 字体定义
% ----------------------------------------------------------

\setCJKmainfont{FZShuSong}
[
  Path           = fonts/,
  BoldFont       = FZHei,
  ItalicFont     = FZKai,
  BoldItalicFont = FZHei,
]
\setCJKsansfont{FZHei}[Path=fonts/]
\setCJKmonofont{HYFangSong}[Path=fonts/]

\setCJKfamilyfont{ss}{FZShuSong}[Path=fonts/]
\setCJKfamilyfont{fs}{FZFangSong}[Path=fonts/]
\setCJKfamilyfont{ht}{FZHei}[Path=fonts/]
\setCJKfamilyfont{kt}{FZKai}[Path=fonts/]
\setCJKfamilyfont{xb}{FZXiaoBiaoSong}[Path=fonts/]
\setCJKfamilyfont{xh}{FZXiHeiI}[Path=fonts/]
\setCJKfamilyfont{sz}{Berthold Baskerville Book}[Path=fonts/]

% 为封面设计的压缩字体
\newCJKfontfamily\coverxb{FZXiaoBiaoSong}[Path=fonts/, FakeStretch=0.85]
\newCJKfontfamily\coverxh{FZXiHeiI}[Path=fonts/, FakeStretch=0.85]

% 定义经过轻微压缩的字体族 (90%)
\newCJKfontfamily\modifiedss{FZShuSong}[Path=fonts/, FakeStretch=0.9]
\newCJKfontfamily\modifiedfs{FZFangSong}[Path=fonts/, FakeStretch=0.9]
\newCJKfontfamily\modifiedht{FZHei}[Path=fonts/, FakeStretch=0.9]
\newCJKfontfamily\modifiedkt{FZKai}[Path=fonts/, FakeStretch=0.9]
\newCJKfontfamily\modifiedxb{FZXiaoBiaoSong}[Path=fonts/, FakeStretch=0.9]
\newCJKfontfamily\modifiedxh{FZXiHeiI}[Path=fonts/, FakeStretch=0.9]

% 定义经过更轻微压缩的字体族 (95%)
\newCJKfontfamily\modifiedmodifiedss{FZShuSong}[Path=fonts/, FakeStretch=0.95]
\newCJKfontfamily\modifiedmodifiedfs{FZFangSong}[Path=fonts/, FakeStretch=0.95]
\newCJKfontfamily\modifiedmodifiedht{FZHei}[Path=fonts/, FakeStretch=0.95]
\newCJKfontfamily\modifiedmodifiedkt{FZKai}[Path=fonts/, FakeStretch=0.95]
\newCJKfontfamily\modifiedmodifiedxb{FZXiaoBiaoSong}[Path=fonts/, FakeStretch=0.95]
\newCJKfontfamily\modifiedmodifiedxh{FZXiHeiI}[Path=fonts/, FakeStretch=0.95]

% 标点符号样式设置
\xeCJKEditPunctStyle{kaiming}
{
  fixed-margin-ratio = 0.000,
  mixed-margin-ratio = 1.000,
  middle-margin-ratio = 0.000,
}

% ----------------------------------------------------------
% 版面与边距
% ----------------------------------------------------------

\geometry{
  paperwidth  = 437bp,
  paperheight = 613bp,
  top         = 94.3bp,
  bottom      = 60.0bp,
  left        = 76.5bp,
  right       = 66.5bp,
  headsep     = 25.0bp,
  footskip    = 17.0bp,
  footnotesep = 11bp,
}

% ----------------------------------------------------------
% 页眉、页脚与页码
% ----------------------------------------------------------

\pagestyle{fancy}

\fancyhf{}

% --------------------------------------
% 页眉与页脚
% --------------------------------------

\renewcommand{\headrulewidth}{0bp}

\fancyhead[RO]{\rightmark}
\fancyhead[LE]{\leftmark}

\renewcommand{\chaptermark}[1]{%
  \sbox{\headerbox}{\small\modifiedxh#1}%
  \ifdim\wd\headerbox > 20em
  \def\theheadermark{\truncate[……]{20em}{\small\modifiedxh#1}}%
  \else
  \def\theheadermark{\small\modifiedxh#1}%
  \fi
  \markboth{\theheadermark}{}%
}

\renewcommand{\sectionmark}[1]{%
  \sbox{\headerbox}{\small\modifiedxh#1}%
  \ifdim\wd\headerbox > 20em
  \def\theheadermark{\truncate[……]{20em}{\small\modifiedxh#1}}%
  \else
  \def\theheadermark{\small\modifiedxh#1}%
  \fi
  \markright{\theheadermark}%
}

% --------------------------------------
% 页码
% --------------------------------------

\fancyfoot[RO]{\fontspec{Berthold Baskerville Book}[Path=fonts/]\thepage}
\fancyfoot[LE]{\fontspec{Berthold Baskerville Book}[Path=fonts/]\thepage}

\assignpagestyle{\part}{empty}
\fancypagestyle{empty}{%
  \fancyhf{}%
  \fancyfoot{}%
}

\fancypagestyle{plain}{%
  \fancyhf{}%
  \fancyfoot[RO]{%
    \makebox[0pt][l]{%
      \hspace{3.5bp}%
      \fontspec{Berthold Baskerville Book}[Path=fonts/]\thepage
    }%
  }%
  \fancyfoot[LE]{\fontspec{Berthold Baskerville Book}[Path=fonts/]\thepage}%
}

% ----------------------------------------------------------
% 注释
% ----------------------------------------------------------

\AtBeginUTFCommand[\textcircled]{\begroup\EnclosedNumbers}
\AtEndUTFCommand[\textcircled]{\endgroup}

\xeCJKDeclareCharClass{Default}{"24EA, "2460->"2473, "3251->"32BF}
\newfontfamily\EnclosedNumbers{FZShuSong}[Path=fonts/]

\makeatletter

\setlength{\skip\footins}{2\baselineskip plus 2\baselineskip minus 2\baselineskip}

% --------------------------------------
% authornote
% --------------------------------------

\DeclareNewFootnote{B}

\newcommand\defaultfootnoterule{\vskip 8.0bp\hrule width 74bp height 0.3bp\vskip 8.0bp}

\let\fn@footnoteB\thefootnoteB
\let\fn@fnfootnoteB\footnoteB

\renewcommand{\thefootnoteB}{%
  \CJKfamily{fs}%
  \fontsize{8.0bp}{8.0bp}\selectfont%
  (\fn@footnoteB)%
}

\newcommand{\authornote}[1]{%
  \deffootnote[5em]{0em}{0em}{%
    \raisebox{0.3bp}{%
      \thefootnotemark\hspace{0.5em}%
    }%
  }%
  \fn@fnfootnoteB{%
    \CJKfamily{fs}%
    \fontsize{9.0bp}{9.5bp}\selectfont%
    #1%
  }%
  \hspace{-0.2em}%
}

% --------------------------------------
% editornote
% --------------------------------------

\DeclareNewFootnote{C}

\renewcommand\extrafootnoterule{\vskip -4.0bp\hrule width \textwidth height 0.3bp\vskip 8.0bp}

\let\fn@fnfootnoteC\footnoteC

\renewcommand\thefootnoteC{%
  \textcircled{%
    \arabic{footnoteC}%
  }%
}

\newcommand{\editornote}[1]{%
  \hspace{0.1em}%
  \deffootnote{2em}{0em}{%
    \raisebox{0.4bp}{%
      \makebox[2em][l]{\thefootnotemark}%
    }%
  }%
  \fn@fnfootnoteC{%
    \CJKfamily{fs}%
    \fontsize{9.0bp}{9.5bp}\selectfont%
    #1%
  }%
}

\MakePerPage{footnoteC}

\makeatother

% --------------------------------------
% endnote
% --------------------------------------

\setenotez{
  list-heading=\chapter*{\CJKfamily{kt}\fontsize{16.5bp}{16.5bp}\selectfont\ziju{1.500}{注释}},
  backref=true
}

\renewcommand\enmark[1]{%
  {\CJKfamily{ss}\fontsize{9.0bp}{9.75bp}\selectfont #1}\hspace{1em}%
}

\let\originalendnote\endnote

\renewcommand{\endnote}[1]{%
  \hspace{0.05em}%
  \originalendnote{%
    \begingroup
    \leftskip=2.9em\relax%
    \rightskip=0.1em\relax%
    \CJKfamily{ss}%
    \fontsize{9.0bp}{9.75bp}\selectfont%
    \ziju{0.050}%
    #1%
    \par%
    \endgroup%
    \vspace{-0.6em}%
  }%
  \hspace{-0.05em}%
}

\fancypagestyle{endnotestyle}{%
  \fancyhf{}%
  \renewcommand{\headrulewidth}{0bp}%

  \fancyhead[RO]{\small\CJKfamily{xh}\scalebox{0.9}[1.0]{\ziju{1.5pt}{注释}}}
  \fancyhead[LE]{\small\CJKfamily{xh}\scalebox{0.9}[1.0]{\ziju{1.5pt}{注释}}}

  \fancyfoot[RO]{%
    \makebox[0pt][l]{%
      \hspace{3.5bp}%
      \fontspec{Berthold Baskerville Book}[Path=fonts/]\thepage
    }%
  }
  \fancyfoot[LE]{\hspace{-4.5bp}\makebox[0pt][r]{\fontspec{Berthold
  Baskerville Book}[Path=fonts/]\thepage}}
}

% --------------------------------------

% ----------------------------------------------------------
% 自定义命令
% ----------------------------------------------------------

% --- 副标题 (vicetitle) ---
\newcommand{\vicetitle}[1]{%
  \vspace{-3.0bp}%
  \begin{center}%
    \begin{minipage}[t]{\textwidth}%
      \CJKfamily{fs}%
      \fontsize{10.5bp}{10.5bp}\selectfont\ziju{0.010}%
      \centering%
      (#1)
    \end{minipage}%
    \vspace{18.5bp}%
  \end{center}%
}

% --- 引言 (quote) ---
\renewenvironment{quote}
{%
  \list{}{\rightmargin=0pt \leftmargin=0pt}%
\item[]\relax\vspace{0.2em}
  \fontsize{9.0bp}{9.75bp}\selectfont
  \hspace*{2em}%
}
{%
  \vspace{-0.2em}\endlist
}

% --- 落款 (closing) ---
\newcommand{\closing}[2]{%
  \vspace{10.5bp}%
  \begin{flushright}%
    \CJKfamily{fs}%
    \fontsize{9.0bp}{9.5bp}\selectfont\ziju{0.010}%
    #1
    \if\relax#2\relax%
    \else%
    \par#2%
    \fi%
  \end{flushright}%
}

% --- 信息栏 (info) ---
\newcommand{\info}[4]{%
  \vspace{-4.0bp}%
  \begin{center}%
    \begin{minipage}[t]{\textwidth}%
      \begin{minipage}[t]{0.40\textwidth}%
        \begin{minipage}[t]{\textwidth}%
          \CJKfamily{ss}%
          \fontsize{9.0bp}{9.5bp}\selectfont\ziju{0.010}%
          #1%
        \end{minipage}%
        \\%
        \raisebox{-10bp}{%
          \begin{minipage}[t]{\textwidth}%
            \CJKfamily{ss}%
            \fontsize{9.0bp}{9.5bp}\selectfont\ziju{0.010}%
            #2%
          \end{minipage}%
        }%
      \end{minipage}%
      \hfill%
      \begin{minipage}[t]{0.40\textwidth}%
        \begin{minipage}[t]{\textwidth}%
          \CJKfamily{ss}%
          \fontsize{9.0bp}{9.5bp}\selectfont\ziju{0.010}%
          #3%
        \end{minipage}%
        \\%
        \raisebox{-10bp}{%
          \begin{minipage}[t]{\textwidth}%
            \CJKfamily{ss}%
            \fontsize{9.0bp}{9.5bp}\selectfont\ziju{0.010}%
            #4%
          \end{minipage}%
        }%
      \end{minipage}%
    \end{minipage}%
  \end{center}%
}

% --- 占位框 (todo) ---
\newcommand{\todo}{%
  \par\vspace{12bp}%
  \noindent%
  \makebox[\textwidth][c]{%
    \begin{tikzpicture}%
      \node[%
        draw,%
        rectangle,%
        line width=1pt,%
        minimum height=5\baselineskip,%
        text width=0.8\textwidth,%
        align=center%
      ]%
      {这里应该添加一张图像或者一段文字};%
    \end{tikzpicture}%
  }%
  \par\vspace{4bp}%
}

% --- 居中盒子 (centerbox) ---
\newcommand{\centerbox}[1]{%
  \par\removelastskip\vspace{\dimexpr\baselineskip - 1em\relax}%
  \noindent%
  \makebox[\textwidth][c]{%
    \begin{minipage}{\textwidth}%
      \centering%
      #1%
    \end{minipage}%
  }%
  \par\removelastskip\vspace{\dimexpr\baselineskip - 1em\relax}%
}

% --- 图像盒子 (imagebox) ---
\newcommand{\imagebox}[1]{%
  \par\vspace{\dimexpr\baselineskip - 1em\relax}%
  \noindent%
  \makebox[\textwidth][c]{%
    \includegraphics[width=\textwidth]{#1}%
  }%
  \par\vspace{\dimexpr\baselineskip - 1.8em\relax}%
}

% --- 字体盒子 (fontbox) ---
\newcommand{\fontbox}[1]{%
  \raisebox{0.15ex}{#1}%
}

% --- 悬挂段落 (indentpara) ---
\newcommand{\indentpara}[1]{%
  \noindent\hangindent=#1 \hangafter=0%
}

% ----------------------------------------------------------
% 目录与标题
% ----------------------------------------------------------

\ctexset{
  tocdepth    = 10,
  secnumdepth = 6,
}

\renewcommand{\contentsname}{%
  \CJKfamily{xb}%
  \fontsize{16.5bp}{16.5bp}\selectfont\ziju{1.500}%
  目录%
}

% --------------------------------------
% part
% --------------------------------------

\titlecontents{part}[0bp]%
{%
  \fontspec{Berthold Baskerville Book}[Path=fonts/]%
  \fontsize{14.5bp}{14.5bp}\selectfont\ziju{0.100}%
  \centering%
  \vspace{7.5bp}%
}%
{}%
{}%
{}[\vspace{5.5bp}]%

\newcommand\partbox{%
  \centering%
  \begin{minipage}{0.80\textwidth}%
    \fontspec{Berthold Baskerville Book}[Path=fonts/]%
    \fontsize{32.0bp}{32.0bp}\selectfont\ziju{0.100}%
    \centering%
  }

\titleformat{\part}[display]%
  {\partbox}{}{0bp}%
  {}[%
\end{minipage}]

\ctexset{
  part / name   = ,
  part / number = \hspace{-1em},
}

\ctexset{
  part / fixskip    = true,
  part / afterskip  = ,
  part / beforeskip = ,
}

% --------------------------------------
% chapter
% --------------------------------------

\titlecontents{chapter}[0bp]%
{%
  \CJKfamily{xb}%
  \fontsize{9.5bp}{9.5bp}\selectfont\ziju{0.150}%
  \vspace{3bp}%
}%
{}%
{}%
{\hspace{3bp}\titlerule*[.4em]{$\cdot$}\contentspage}

\newcommand\chapterbox{%
  \centering%
  \begin{minipage}{0.80\textwidth}%
    \modifiedmodifiedxb%
    \fontsize{16.5bp}{18.0bp}\selectfont\ziju{0.050}%
    \centering%
  }

\titleformat{\chapter}[display]%
  {\chapterbox}{}{0bp}%
  {}[%
\end{minipage}]

\titlespacing{\chapter}%
{0bp}%
{0bp}%
{18.0bp}%

\ctexset{
  chapter / fixskip    = true,
  chapter / afterskip  = ,
  chapter / beforeskip = ,
}

% --------------------------------------
% chapterx
% --------------------------------------

\titleclass{\chapterx}{straight}[\part]

\newcounter{chapterx}[part]
\renewcommand\thechapterx{\hspace{-1em}}

\titlecontents{chapterx}[0bp]%
{%
  \CJKfamily{fs}%
  \fontsize{10.5bp}{10.5bp}\selectfont\ziju{0.050}%
  \vspace{3bp}%
}%
{}%
{}%
{\hspace{3bp}\titlerule*[.4em]{$\cdot$}\contentspage}

\newcommand\chapterxbox{%
  \centering%
  \begin{minipage}{0.80\textwidth}%
    \modifiedmodifiedxb%
    \fontsize{16.5bp}{18.0bp}\selectfont\ziju{0.050}%
    \centering%
  }

\titleformat{\chapterx}[display]%
  {\chapterxbox}{}{0bp}%
  {}[%
\end{minipage}]

\titlespacing{\chapterx}%
{0bp}%
{0bp}%
{18.0bp}%

% --------------------------------------
% section
% --------------------------------------

\titlecontents{section}[2em]%
{%
  \CJKfamily{fs}%
  \fontsize{9.5bp}{9.5bp}\selectfont\ziju{-0.050}%
  \vspace{3bp}%
}%
{\contentspush{\thecontentslabel\hspace{1em}}}%
{}%
{\hspace{3bp}\titlerule*[.4em]{$\cdot$}\contentspage}

\newcommand\sectionbox{%
  \centering%
  \begin{minipage}{0.50\textwidth}%
    \modifiedmodifiedxh%
    \fontsize{14.5bp}{15.5bp}\selectfont\ziju{0.025}%
    \centering%
  }

\titleformat{\section}[display]%
  {\sectionbox}%
  {\ziju{0.5}第\chinese{section}篇}%
  {5.0bp}%
  {}%
  [%
\end{minipage}]

\ctexset{
  section / name   = {第,篇},
  section / number = \chinese{section},
}

\ctexset{
  section / fixskip    = true,
  section / afterskip  = ,
  section / beforeskip = ,
}

\titlespacing{\section}%
{0bp}%
{28.0bp}%
{26.0bp}%

% --------------------------------------
% subsection
% --------------------------------------

\titlecontents{subsection}[2em]%
{%
  \CJKfamily{ss}%
  \fontsize{9.0bp}{9.0bp}\selectfont\ziju{0.010}%
  \vspace{3bp}%
}%
{\contentspush{\thecontentslabel\hspace{1em}}}%
{}%
{\hspace{3bp}\titlerule*[.4em]{$\cdot$}\contentspage}

\newcommand\subsectionbox{%
  \centering%
  \begin{minipage}{0.70\textwidth}%
    \modifiedmodifiedss%
    \fontsize{14.0bp}{15.0bp}\selectfont\ziju{0.025}%
    \centering%
  }

\titleformat{\subsection}[display]%
  {\subsectionbox}%
  {\ziju{0.5}第\chinese{subsection}章}%
  {5.0bp}%
  {}%
  [%
\end{minipage}]

\ctexset{
  subsection / name   = {第,章},
  subsection / number = \chinese{subsection},
}

\ctexset{
  subsection / fixskip    = true,
  subsection / afterskip  = ,
  subsection / beforeskip = ,
}

\titlespacing{\subsection}%
{0bp}%
{14.5bp}%
{13.0bp}%

% --------------------------------------
% subsubsection
% --------------------------------------

\titlecontents{subsubsection}[4em]%
{%
  \CJKfamily{ss}%
  \fontsize{9.0bp}{9.0bp}\selectfont\ziju{0.010}%
  \vspace{3bp}%
}%
{\contentspush{\thecontentslabel\hspace{1em}}}%
{}%
{\hspace{3bp}\titlerule*[.4em]{$\cdot$}\contentspage}

\newcommand\subsubsectionbox[1]{%
  \begin{center}%
    \begin{minipage}{0.60\textwidth}%
      \modifiedmodifiedss%
      \fontsize{12.5bp}{14.5bp}\selectfont\ziju{0.025}%
      \centering%
      #1%
    \end{minipage}%
  \end{center}%
}

\ctexset{
  subsubsection / name       = {,.},
  subsubsection / number     = \arabic{subsubsection},
  subsubsection / format     = \subsubsectionbox,
  subsubsection / aftername  = ,
}

\ctexset{
  subsubsection / fixskip    = true,
  subsubsection / afterskip  = ,
  subsubsection / beforeskip = ,
}

\titlespacing{\subsubsection}%
{0bp}%
{19.5bp}%
{9.0bp}%

% --------------------------------------
% paragraph
% --------------------------------------

\titlecontents{paragraph}[6em]%
{%
  \CJKfamily{ss}%
  \fontsize{9.0bp}{9.0bp}\selectfont\ziju{0.010}%
  \vspace{3bp}%
}%
{\contentspush{\thecontentslabel\hspace{1em}}}%
{}%
{\hspace{3bp}\titlerule*[.4em]{$\cdot$}\contentspage}

\newcommand\paragraphbox[1]{%
  \begin{center}%
    \begin{minipage}{0.80\textwidth}%
      \modifiedmodifiedss%
      \fontsize{12.5bp}{12.5bp}\selectfont\ziju{0.010}%
      \centering%
      #1%
    \end{minipage}%
  \end{center}%
}

\ctexset{
  paragraph / name       = {,.},
  paragraph / number     = \Alph{paragraph},
  paragraph / format     = \paragraphbox,
  paragraph / aftername  = ,
}

\ctexset{
  paragraph / fixskip    = true,
  paragraph / afterskip  = ,
  paragraph / beforeskip = ,
}

\titlespacing{\paragraph}%
{0bp}%
{18.0bp}%
{9.0bp}%

% --------------------------------------
% subparagraph
% --------------------------------------

\titlecontents{subparagraph}[8em]%
{%
  \CJKfamily{ss}%
  \fontsize{9.0bp}{9.0bp}\selectfont\ziju{0.010}%
  \vspace{3bp}%
}%
{\contentspush{\thecontentslabel\hspace{1em}}}%
{}%
{\hspace{3bp}\titlerule*[.4em]{$\cdot$}\contentspage}

\newcommand\subparagraphbox[1]{%
  \begin{center}%
    \begin{minipage}{0.80\textwidth}%
      \CJKfamily{ht}%
      \fontsize{10.5bp}{10.5bp}\selectfont\ziju{0.010}%
      \centering%
      #1%
    \end{minipage}%
  \end{center}%
}

\ctexset{
  subparagraph / name         = {\CJKfamily{ss}(,\CJKfamily{ss})},
  subparagraph / number       = \arabic{subparagraph},
  subparagraph / format       = \subparagraphbox,
  subparagraph / numberformat = \bf,
  subparagraph / aftername    = ,
}

\ctexset{
  subparagraph / fixskip      = true,
  subparagraph / afterskip    = ,
  subparagraph / beforeskip   = ,
}

\titlespacing{\subparagraph}%
{0bp}%
{0bp}%
{-1bp}%

% ------------------------------------------------------------------------------
% 重命名
%
% \tyear          -> 年份标题 (可选)
% \tdoc           -> 文献标题
% \tpart          -> 篇 (可选)
% \tchapter       -> 章
% \tsection       -> 节 (如: 1.)
% \tsubsection    -> 子节 (如: A.)
% \tsubsubsection -> 小节 (如: (1))
% ------------------------------------------------------------------------------

\newcommand{\tyear}[1]{%
  \cleardoublepage%
  \thispagestyle{empty}%
  \part{#1}\label{#1}%
}

\newcommand{\tdoc}[1]{%
  \cleardoublepage%
  \thispagestyle{empty}%
  \let\oldthispagestyle\thispagestyle%
  \renewcommand{\thispagestyle}[1]{}%
  \chapter{#1}\label{#1}%
  \let\thispagestyle\oldthispagestyle%
}

\newcommand{\tpart}[1]{%
  \cleardoublepage%
  \section{#1}\label{#1}%
}

\newcommand{\tpartnonum}[1]{%
  \cleardoublepage%
  \section*{#1}\label{#1}%
  \addcontentsline{toc}{section}{#1}%
  \partmark{#1}%
}

\newcommand{\tchapter}[1]{%
  \clearpage%
  \subsection{#1}\label{#1}%
}

\newcommand{\tchapternonum}[1]{%
  \clearpage%
  \subsection*{#1}\label{#1}%
  \addcontentsline{toc}{subsection}{#1}%
  \chaptermark{#1}%
}

\newcommand{\tsection}[1]{%
  \subsubsection{#1}\label{#1}%
}

\newcommand{\tsectionnonum}[1]{%
  \subsubsection*{#1}\label{#1}%
  \addcontentsline{to}{subsubsection}{#1}%
  \sectionmark{#1}%
}

\newcommand{\tsubsection}[1]{%
  \paragraph{#1}\label{#1}%
}

\newcommand{\tsubsectionnonum}[1]{%
  \paragraph*{#1}\label{#1}%
  \addcontentsline{toc}{paragraph}{#1}%
}

\newcommand{\tsubsubsection}[1]{%
  \subparagraph{#1}\label{#1}%
}

\newcommand{\tsubsubsectionnonum}[1]{%
  \subparagraph*{#1}\label{#1}%
  \addcontentsline{toc}{subparagraph}{#1}%
}


% ------------------------------------------------------------------------------
% 正文
% ------------------------------------------------------------------------------

\begin{document}

% --------------------------------------

% ------------------
% 封面
% ------------------

\cleardoublepage%
\thispagestyle{empty}%
{%
  \vspace*{\fill}%
  \begin{center}%
    \vspace{-12.5em}
    {\CJKfamily{coverxb}\fontsize{38bp}{47.5bp}\selectfont\ziju{0.200} 模版样本\par}
  \end{center}
  \vspace*{\fill}%
}
\cleardoublepage%

% ------------------
% 宣言
% ------------------

\cleardoublepage%
\thispagestyle{empty}%
\definecolor{deepred}{rgb}{0.6, 0, 0}
{\color{deepred}\CJKfamily{xb}\fontsize{14bp}{14.5bp}\selectfont\ziju{0.100}%
  \center 这是一段题词,可以是名言、格言或其他内容。
\par}

\frontmatter
\pagenumbering{arabic}


% --------------------------------------

\xeCJKsetup{CJKglue={\hskip 0bp plus 0.02\baselineskip minus 0.000\baselineskip}}
\setlength{\parskip}{0bp}

\clearpage
\fancyhf{}
\renewcommand{\headrulewidth}{0bp}

\fancyhead[RO]{\small\CJKfamily{xh}\scalebox{0.9}[1.0]{\ziju{1.5pt}{目录}}}
\fancyhead[LE]{\small\CJKfamily{xh}\scalebox{0.9}[1.0]{\ziju{1.5pt}{目录}}}
\fancyfoot[RO]{%
  \makebox[0pt][l]{%
    \hspace{3.5bp}%
    \fontspec{Berthold Baskerville Book}[Path=fonts/]\thepage
  }%
}
\fancyfoot[LE]{\hspace{-4.5bp}\makebox[0pt][r]{\fontspec{Berthold Baskerville Book}[Path=fonts/]\thepage}}
\pagestyle{fancy}

\tableofcontents

\mainmatter

\clearpage
\fancyhf{}
\renewcommand{\headrulewidth}{0bp}

\fancyhead[RO]{\rightmark}
\fancyhead[LE]{\leftmark}
\fancyfoot[RO]{%
  \makebox[0pt][l]{%
    \hspace{3.5bp}%
    \fontspec{Berthold Baskerville Book}[Path=fonts/]\thepage
  }%
}
\fancyfoot[LE]{\hspace{-4.5bp}\makebox[0pt][r]{\fontspec{Berthold Baskerville Book}[Path=fonts/]\thepage}}
\pagestyle{fancy}

\markboth{}{}

% --------------------------------------

\tdoc{《剩余价值理论》}

\tpart{《剩余价值理论》第一册}

\tchapter{[《剩余价值理论》手稿目录]}

\indentpara{0em}[\endnote{《剩余价值理论》是马克思的主要著作《资本论》的第四卷。马克思把《资本论》的前三卷称为理论部分,把第四卷称为历史部分、历史批判部分或历史文献部分。在这一卷中,马克思围绕着剩余价值理论这个政治经济学的核心问题,对各派资产阶级经济学家的理论进行了系统的、历史的分析批判,同时以论战的形式阐述了自己的政治经济学理论的许多重要方面。马克思从十九世纪四十年代起就开始研究政治经济学,并计划写一部批判现存制度和资产阶级政治经济学的巨著。经过长期系统研究,于 1857—1858 年写了一部经济学手稿,在这个手稿的基础上于 1859 年出版了《政治经济学批判》(第一分册)。从 1861 年 8 月到 1863 年 7 月,又写了一部篇幅很大的手稿。这部手稿的大部分,也是整理得最细致的部分,构成《剩余价值理论》。手稿的其余部分,即理论部分,后来经马克思重新修改和补充,形成了《资本论》前三卷的内容,而《剩余价值理论》这一历史部分没有重新加工,仍保持着原来的样子。马克思生前出版了《资本论》第一卷。他逝世后,恩格斯整理出版了《资本论》第二卷和第三卷,但是没有来得及整理出版《资本论》第四卷即《剩余价值理论》。1905—1910 年卡尔·考茨基编辑出版了《剩余价值理论》,他对马克思的手稿做了许多删改和变动。1954—1961 年按马克思的手稿次序编辑出版了《剩余价值理论》俄文新版本;1956—1962 年出版了该书德文新版本;1962—1964 年则作为《马克思恩格斯全集》俄文第二版第二十六卷(共三册)出版。《剩余价值理论》的章节标题大部分是由俄文版编者拟定的。编者加的标题和文字,用方括号[]标出。马克思手稿中使用的方括号则改用花括号\fontbox{~\{} \fontbox{\}~}。马克思手稿的稿本编号和页码,一律用方括号标出,括号中的罗马数字表示稿本编号,阿拉伯数字表示页码。——第 1 页。}VI—219b]第 VI 本目录:\endnote{《剩余价值理论》手稿的这一目录是马克思写在 1861—1863 年手稿第 VI—XV 各稿本的封面上的。其中有几本的目录比正文先写,这从后来写完相应稿本的正文时对目录所作的修改和删节中可以看出。第 XIV 本封面上的目录并不符合稿本的实际内容:这一目录是第 XIV、XV 和 XVIII 本中所完成的计划。——第 3 页。}

% \editornote{为了方便,这里节选《剩余价值理论》的部分内容作为模版的正文。这种注释一般是编者加的。}

\indentpara{2em}(5)剩余价值理论\endnote{马克思在《剩余价值理论》这一标题之前写了阿拉伯数字“5”。这表示关于资本的研究的第一章第五节,即最后一节;马克思打算把这一研究作为论述商品和货币的《政治经济学批判》第一分册的直接继续来发表。在这第五节之前,在手稿第 I—V 本中有三节概述:(1)货币转化为资本,(2)绝对剩余价值,(3)相对剩余价值。在第 V 本第 184 页,马克思指出:“在相对剩余价值之后,应当考察绝对剩余价值和相对剩余价值两者的结合。”这一考察本应构成第四节,但当时并未写成。马克思还没有写完第三节就马上转写第五节,即《剩余价值理论》。——第 3 页。}

\indentpara{4em}(a)\authornote{这种注释一般是作者加的。}詹姆斯·斯图亚特爵士

\indentpara{4em}(b)重农学派

\indentpara{4em}(c)亚·斯密[VI—219b]

\indentpara{0em}[VII—272b][第 VII 本目录]

\indentpara{2em}(5)剩余价值理论

\indentpara{4em}(c)亚·斯密(续篇)

\indentpara{6em}(研究年利润和年工资怎样才能购买一年内生产的、除利润和工资外还包括不变资本的商品)[VII—272b]

\indentpara{0em}[VIII—331b][第 VIII 本目录]

\indentpara{2em}(5)剩余价值理论

\indentpara{4em}(c)亚·斯密(结尾)\endnote{实际上这并不是论斯密这一节的“结尾”,而只是“续篇”。这一节的结尾部分在下一本即手稿第 IX 本中。——第 3 页。}[VIII—331b]

\indentpara{0em}[IX—376b][第 IX 本目录]

\indentpara{2em}(5)剩余价值理论

\indentpara{4em}(c)亚·斯密。结尾

\indentpara{4em}(d)奈克尔[IX—376b]

\indentpara{0em}[X—421c][第 X 本目录]

\indentpara{2em}(5)剩余价值理论

\indentpara{6em}插入部分。魁奈的经济表

\indentpara{4em}(e)兰盖

\indentpara{4em}(f)布雷

\indentpara{4em}(g)洛贝尔图斯先生。插入部分。新的地租理论[X—421c]

\indentpara{0em}[XI—490a][第 XI 本目录]

\indentpara{2em}(5)剩余价值理论

\indentpara{4em}(g)洛贝尔图斯

\indentpara{6em}插入部分。评所谓李嘉图规律的发现史

\indentpara{4em}(h)李嘉图

\indentpara{6em}李嘉图和亚·斯密的费用价格理论(批驳部分)

\indentpara{6em}李嘉图的地租理论

\indentpara{6em}级差地租表及其说明[XI—490a]

\indentpara{0em}[XII—580b][第 XII 本目录]

\indentpara{2em}(5)剩余价值理论

\indentpara{4em}(h)李嘉图

\indentpara{6em}级差地租表及其说明

\indentpara{6em}(考察生活资料和原料的价值——以及机器的价值——的变动对资本有机构成的影响)

\indentpara{6em}李嘉图的地租理论

\indentpara{6em}亚·斯密的地租理论

\indentpara{6em}李嘉图的剩余价值理论

\indentpara{6em}李嘉图的利润理论[XII—580b]

\indentpara{0em}[XIII—670a][第 XIII 本目录]

\indentpara{2em}(5)剩余价值理论及其他

\indentpara{4em}(h)李嘉图

\indentpara{6em}李嘉图的利润理论

\indentpara{6em}李嘉图的积累理论。对这个理论的批判。(从资本的基本形式得出危机)

\indentpara{6em}李嘉图的其他方面。论李嘉图这一节的结尾(约翰·巴顿)

\indentpara{4em}(i)马尔萨斯[XIII—670a]

\indentpara{0em}[XIV—771a][第 XIV 本目录和《剩余价值理论》最后几章的计划]

\indentpara{2em}(5)剩余价值理论

\indentpara{4em}(i)马尔萨斯

\indentpara{4em}(k)李嘉图学派的解体(托伦斯、詹姆斯·穆勒、普雷沃、几部论战著作、麦克库洛赫、威克菲尔德、斯特林、约·斯·穆勒)

\indentpara{4em}(l)政治经济学家的反对派\endnote{关于政治经济学家的反对派一章在手稿第 XIV 本中仅仅是开始。这一章的续篇在第 XV 本前半部分。——第 5 页。}

\indentpara{6em}(政治经济学家的反对派布雷)\endnote{布雷《劳动中的不公正现象及其消除办法》一书的摘录和马克思所加的为数不多的评语包含在手稿第 X 本中。——第 5 页。}

\indentpara{4em}(m)拉姆赛

\indentpara{4em}(n)舍尔比利埃

\indentpara{4em}(o)理查·琼斯。\endnote{论拉姆赛、舍尔比利埃和理·琼斯的几章包含在手稿第 XVIII 本中。——第 5 页。}(这第五部分结束)

\indentpara{6em}补充部分:收入及其源泉\endnote{马克思在手稿第 XV 本后半部分论述了收入及其源泉,在这方面揭示了庸俗政治经济学的阶级根源和认识论根源。这个“补充部分”马克思后来决定放在《资本论》第三部分,这从他在 1863 年 1 月拟定的这一部分的计划可以看出;按照这一计划,第九章的标题应该是《收入及其源泉》(见本册第 447 页)。——第 5 页。}[XIV—771a]

\indentpara{0em}[XV—862a][第 XV 本目录]

\indentpara{2em}(5)剩余价值理论

\indentpara{4em}(1)以李嘉图理论为依据的无产阶级反对派

\indentpara{4em}(2)莱文斯顿。结尾\endnote{论莱文斯顿一节是在前一本(手稿第 XIV 本)第 861 页开始的。第 XIV 本中在这一节之前有标以数码“1”的一节,即论匿名小册子《根据政治经济学基本原理得出的国民困难的原因及其解决办法》。——第 6 页。}

\indentpara{4em}(3)和(4)霍吉斯金\endnote{论霍吉斯金一节的结尾包含在手稿第 XVIII 本中(第 1084—1086 页)。——第 6 页。}

\indentpara{4em}(现存财富同生产运动的关系)

\indentpara{4em}所谓积累不过是流通现象(储备等等是流通的蓄水池)

\indentpara{4em}(复利;根据复利说明利润率的下降)

\indentpara{6em}庸俗政治经济学\endnote{马克思在手稿第 XV 本中研究收入及其源泉的问题时,对庸俗政治经济学进行了分析。他在这一本第 935 页注明参看“论庸俗经济学家一节”,即他的著作中尚未完成的一章,说他在这一章里“将回过头来谈”顺便涉及的蒲鲁东和巴师夏之间的论战。这一提示表明马克思打算专门写一章来批判庸俗政治经济学,但是没有写成。手稿第 XVIII 本中,马克思在结束对霍吉斯金观点的分析并提到后者对资产阶级辩护士的理论的反驳时,注明:“要在论庸俗经济学家一章中谈这一点”(第 1086 页)。这句话也证明马克思打算以后专门写一章来论述庸俗政治经济学。在 1863 年 1 月拟定的《资本论》第三部分的计划中,倒数第二章即第十一章的标题是《庸俗政治经济学》(见本册第 447 页)。——第 6 页。}

\indentpara{4em}(生息资本在资本主义生产基础上的发展)

\indentpara{4em}(生息资本和商业资本同产业资本的关系。更为古老的形式。派生的形式)

\indentpara{4em}(高利贷。路德等等)\endnote{马克思在手稿第 XV 本封面上写下了这一本的目录,目录中的某些标题是写在旁边或上面的。在本卷发表的目录中,这些标题是按照符合稿本实际内容的次序排列的。——第 6 页。}[XV—862a]

\tchapter{[总的评论]}

[VI—220]\CJKunderdot{所有经济学家}都犯了一个错误:他们不是就剩余价值的纯粹形式,不是就剩余价值本身,而是就利润和地租这些特殊形式来考察剩余价值。由此必然会产生哪些理论谬误,这将在第三章中得到更充分的揭示,那里要分析以利润形式出现的剩余价值所采取的完全转化了的形式。\endnote{马克思这里说的“第三章”是指关于“资本一般”的研究的第三部分。这一章的标题应为:《资本的生产过程和流通过程的统一,或资本和利润》。以后(例如,见第 IX 本第 398 页和第 XI 本第 526 页)马克思不用“第三章”而用“第三篇”(《dritterAbschnitt》)。后来他就把这第三章称作“第三册”(例如,在 1865 年 7 月 31 日给恩格斯的信中)。关于“资本一般”的研究的“第三章”马克思是在第 XVI 本开始的。从这“第三章”或“第三篇”的计划草稿(见本册第 447 页)中可以看出,马克思打算在那里写两篇专门关于利润理论的历史补充部分。但是马克思在写作《剩余价值理论》的过程中,就已在自己的这一历史批判研究的范围内,详细地批判分析了各种资产阶级经济学家对利润的看法。因此,马克思在《剩余价值理论》中,特别是在这一著作的第二册和第三册中,就已进一步更充分地揭示了由于把剩余价值和利润混淆起来而产生的理论谬误。——第 7、87、272 页。}

\closing{作者}{2025 年 8 月 20 日}

\info{作者写于 2025 年 8 月 20 日}{载于 GitHub}{原文是中文}{选自中文马克思主义文库}

\tpart{《剩余价值理论》第二册}

\tchapter{[第一章]詹姆斯·斯图亚特爵士}

\vicetitle{[区分“让渡利润”和财富的绝对增加]}

\tsection{这是一个用于展示的数字加点标题}

在重农学派以前,剩余价值——即利润,利润形式的剩余价值——完全是用\textbf{交换},用商品高于它的价值出卖来解释的。詹姆斯·斯图亚特爵士,总的说来,并没有超出这种狭隘看法;甚至可以更确切地说,正是斯图亚特科学地复制了这种看法。我说:“科学地”复制。因为斯图亚特不同意这种幻想:单个资本家由于商品高于它的价值出卖而获得的剩余价值,就是新财富的创造。因此,他把\textbf{绝对}利润和\textbf{相对}利润区分开来:

\begin{quote}“\textbf{绝对利润}对谁都不意味着亏损;它是劳动、勤勉或技能的\textbf{增进}的结果,它能引起\textbf{社会财富}的扩大或增加……\textbf{相对利润}对有的人意味着亏损;它表示财富的天平在有关双方之间的摆动,但并不意味着\textbf{总基金的任何增加}……\textbf{混合}利润很容易理解:这种利润……一部分是\textbf{相对的},一部分是\textbf{绝对的}……二者能够不可分割地存在于同一交易中。”(《政治经济学原理研究》,由其子詹姆斯·斯图亚特爵士将军汇编的《詹姆斯·斯图亚特爵士著作集》(六卷集)第 1 卷,1805 年伦敦版第 275—276 页)\end{quote}

\textbf{绝对}利润是由“劳动、勤勉和技能的增进”产生的。究竟它怎样由这种增进产生,斯图亚特并没有想弄清楚。他所加的关于这个利润能引起“\textbf{社会财富}”的扩大和增加的这句话,看来,可以使人得出这样的结论:斯图亚特所指的,仅仅是由劳动生产力的发展造成的使用价值量的增加,他完全离开总是以交换价值的增加为前提的资本家的利润来考察这个绝对利润。这样的解释完全被他进一步的叙述证实了。

\tsubsection{这是一个用于展示的字母标题}

他是这样说的:

\begin{quote}“在商品的\textbf{价格}中,我认为有两个东西是实际存在而又彼此\textbf{完全不同}的:商品的\textbf{实际价值}和\textbf{让渡利润}。”(第 244 页)\end{quote}

可见,商品的价格包含着两个彼此完全不同的要素:第一,商品的\textbf{实际价值};第二,“\textbf{让渡利润}”,即让出或卖出商品时实现的利润。

[221]因此,这个“\textbf{让渡利润}”是由于商品的价格高于商品的实际价值而产生的,换句话说,是由于商品\textbf{高于}它的价值出卖而产生的。这里,一方的赢利总是意味着另一方的亏损。不会造成“总基金的增加”。利润——应该说是剩余价值——是相对的,并且归结为“财富的天平在有关双方之间的摆动”。斯图亚特自己舍弃可以用这种办法来说明剩余价值的看法。他的关于“财富的天平在有关双方之间的摆动”的理论,虽然丝毫没有触及剩余价值本身的性质和起源问题,但是对于考察剩余价值在不同阶级之间按利润、利息、地租这些不同项目进行的分配,有重要的意义。

\tsubsubsection{这是一个用于展示的括号数字标题}

从下面的引文中可以看出,斯图亚特认为,单个资本家的全部利润只限于这种“相对利润”,“让渡利润”。

\begin{quote}他说:“实际价值”决定于“该国一个劳动者平常……在一天、一周、一月……平均能够完成的”劳动“量”。第二,决定于“劳动者用以满足他个人的需要和……购置适合于他的职业的工具的生存资料和必要费用的价值;这些同样也必须平均计算”……第三,决定于“材料的价值”。(第 244—245 页)“如果这三项是已知的,产品的价格就确定了。它不能低于这三项的总和,即不能低于\textbf{实际价值。凡是超过实际价值的,就是厂主的利润}。这个利润将始终同\textbf{需求}成比例,因此它将随情况而变动。”(同上,第 245 页)“由此看来,为了促进制造业的繁荣,必须有大规模的需求……工业家是按照他们有把握取得的利润,来安排自己的开支和自己的生活方式的。”(同上,第 246 页)\end{quote}

从这里可以清楚地看出,“厂主”即单个资本家的利润,总是“相对利润”,总是“让渡利润”,总是由于商品的价格高于商品的实际价值,由于\textbf{商品高于它的价值出卖}而产生的。因此,如果一切商品都按它的\textbf{价值}出卖,那就不会有任何利润了。

关于这个问题,斯图亚特写了专门的一章,他详细地研究

\begin{quote}“利润怎样同生产费用结成一体”。(同上,第 3 卷第 11 页及以下各页)\end{quote}

一方面,斯图亚特抛弃了货币主义和重商主义体系的这样一种看法,即认为商品高于它的价值出卖以及由此产生的利润,形成剩余价值,造成财富的绝对增加;\authornote{其实,连货币主义也认为,这个利润不是在国内产生,而只是在同其他国家的交换中产生。重商主义体系只看到,这个价值表现为货币(金和银),因此剩余价值表现为用货币结算的贸易差额。}另一方面,他仍然维护它们的这样一种观点,即单个资本家的利润无非是价格超过[222]价值的这个余额——“让渡利润”,不过按照他的意见,这种利润只是\textbf{相对的},一方的赢利相当于另一方的亏损,因此,利润的运动归结为“财富的天平在有关双方之间的摆动”。

可见,在这方面,斯图亚特是货币主义和重商主义体系的\textbf{合理的}表达者。

在对资本的理解方面,他的功绩在于:他指出了生产条件作为一定阶级的财产同劳动能力\endnote{原文是:《Arbeitsvermögen》(“劳动能力”)。马克思在 1861—1863 年手稿中在绝大多数场合都用《Arbeitsvermögen》(“劳动能力”)这个术语,而没有用《Arbeitskraft》(“劳动力”)这个术语。在《资本论》第一卷中,马克思把这两个术语当作同一概念使用:“我们把劳动力或劳动能力,理解为人的身体即活的人体中存在的、每当人生产某种使用价值时就运用的体力和智力的总和。”(见马克思《资本论》第 1 卷第 4 章第 3 节)——第 13 页。}分离的过程是怎样发生的。斯图亚特十分注意资本的这个\textbf{产生过程};诚然,他还没有把这个过程直接理解为资本的产生过程,但是,他仍然把这个过程看成是大工业存在的条件。斯图亚特特别在农业中考察了这个过程,并且正确地认为,只是因为农业中发生了这个分离过程,真正的制造业才产生出来。在亚·斯密的著作里,是以这个分离过程已经完成为前提的。

(斯图亚特的书于 1767 年在伦敦[出版],\textbf{杜尔哥}的书[写于]1766 年,亚当·斯密的书——1775 年。)

\tchapter{[第二章]重农学派}

\tsection{[(1)把剩余价值的起源问题从流通领域转到生产领域。把地租看成剩余价值的唯一形式]}

重农学派的重大功绩在于,他们在资产阶级视野以内对\textbf{资本}进行了分析。正是这个功绩,使他们成为现代政治经济学的真正鼻祖。首先,他们分析了资本在劳动过程中借以存在并分解成的各种\textbf{物质组成部分}。决不能责备重农学派,说他们和他们所有的后继者一样,把资本存在的这些物质形式——工具、原料等等,当作跟它们在资本主义生产中出现时的社会条件脱离的资本来理解,简言之,不管劳动过程的社会形式如何,只从它们是一般劳动过程的要素这个形式来理解;从而,把生产的资本主义形式变成生产的一种永恒的自然形式。对于他们来说,生产的资产阶级形式必然以生产的自然形式出现。重农学派的巨大功绩是,他们把这些形式看成社会的生理形式,即从生产本身的自然必然性产生的,不以意志、政策等等为转移的形式。这是物质规律;错误只在于,他们把社会的一个特定历史阶段的物质规律看成同样支配着一切社会形式的抽象规律。

除了对资本在劳动过程中借以组成的物质要素进行这种分析以外,重农学派还研究了资本在流通中所采取的形式(固定资本、流动资本,不过重农学派用的是别的术语),并且一般地确定了资本的流通过程和再生产过程之间的联系。这一点在论流通那一章\endnote{指关于“资本一般”的研究的第二章,这一章最后发展成为《资本论》第二卷。《资本论》第二卷第十章(《关于固定资本和流动资本的理论。重农学派和亚当·斯密》)包含对重农学派关于固定资本和流动资本的观点的分析。在《社会总资本的再生产和流通》一篇的第十九章《前人对这个问题的阐述》中有专门论重农学派的一节。——第 16 页。}再谈。

在这两个要点上,亚·斯密继承了重农学派的遗产。他的功绩,在这方面,不过是把抽象范畴固定下来,对重农学派所分析的差别采用了更稳定的名称。

[223]我们已经看到\endnote{马克思指他的 1861—1863 年手稿第 II 本第 58—60 页(《货币转化为资本》一节,《转化过程的两个组成部分》一小节)。——第 16 页。},资本主义生产发展的基础,一般说来,是\textbf{劳动能力}这种属于工人的\textbf{商品}同劳动条件这种固着于资本形式并脱离工人而独立存在的商品相对立。劳动能力作为商品,它的\textbf{价值}规定具有极重要的意义。这个价值等于把再生产劳动能力所必需的生活资料创造出来的劳动时间,或者说,等于工人作为工人生存所必需的生活资料的价格。只有在这个基础上,才出现劳动能力的\textbf{价值}和这个劳动能力\textbf{所创造的价值}之间的差额,——任何别的商品都没有这个差额,因为任何别的商品的使用价值,从而它的使用,都不能提高它的\textbf{交换价值}或提高从它得到的交换价值。

因此,从事分析资本主义生产的现代政治经济学的基础,就是把\textbf{劳动能力的价值}看作某种固定的东西,已知的量,而实际上它在每一个特定的场合,也就是一个已知量。所以,\textbf{最低限度的工资}理所当然地构成重农学派的学说的轴心。虽然他们还不了解价值本身的性质,他们却能够确定最低限度的工资的概念,这是因为这个\textbf{劳动能力的价值}表现为必要生活资料的价格,因而表现为一定使用价值的总和。他们尽管没有弄清一般价值的性质,但仍然能够在他们的研究所必需的范围内,把劳动能力的价值理解为一定的量。其次,如果说,他们错误地把这个\textbf{最低限度}看作不变的量,在他们看来,这个量完全决定于自然,而不决定于本身就是一个变量的历史发展阶段,那末,这丝毫也不影响他们的结论的抽象正确性,因为劳动能力的价值和这个劳动能力所创造的价值之间的差额,同我们假定劳动能力的价值是大是小毫无关系。

重农学派把关于剩余价值起源的研究从流通领域转到直接生产领域,这样就为分析资本主义生产奠定了基础。

他们完全正确地提出了这样一个基本论点:只有创造\textbf{剩余价值}的劳动,即只有劳动产品中包含的价值超过生产该产品时消费的价值总和的那种劳动,才是\textbf{生产的}。既然原料和材料的价值是已知的,劳动能力的价值又等于最低限度的工资,那末很明显,这个剩余价值只能由工人向资本家提供的劳动超过工人以工资形式得到的劳动量的余额构成。当然,在重农学派那里,剩余价值还不是以这种形式出现的,因为他们还没有把一般价值归结为它的简单实体:劳动量,或劳动时间。

[224]自然,重农学派的表述方式必然决定于他们对价值性质的一般看法,按照他们的理解,价值不是人的活动(劳动)的一定的社会存在方式,而是由土地即自然所提供的物质以及这个物质的各种变态构成的。

劳动能力的\textbf{价值}和这个劳动能力\textbf{所创造的价值}之间的差额,也就是劳动能力使用者由于购买劳动能力而取得的剩余价值,无论在哪个\textbf{生产部门}都不如在\textbf{农业}这个最初的生产部门表现得这样显而易见,这样无可争辩。劳动者逐年消费的生活资料总量,或者说,他消费的物质总量,小于他所生产的生活资料总量。在工业中,一般既不能直接看到工人生产自己的生活资料,也不能直接看到他还生产超过这个生活资料的余额。在这里,过程以买卖为中介,以各种流通行为为中介,而要理解这个过程,就必须分析价值。在农业中,过程在生产出的使用价值超过劳动者消费的使用价值的余额上直接表现出来,因此,不分析价值,不弄清价值的性质,也能够理解这个过程。因此,在把价值归结为使用价值,又把使用价值归结为一般物质的情况下,也能够理解这个过程。所以在重农学派看来,农业劳动是唯一的\textbf{生产劳动},因为按照他们的意见,这是唯一\textbf{创造剩余价值}的劳动,而\textbf{地租}是他们所知道的\textbf{剩余价值的唯一形式}。他们认为,在工业中,工人并不增加物质的量:他只改变物质的形式。材料——物质总量——是农业供给他的。他诚然把价值加到物质上,但这不是靠他的劳动,而是靠他的劳动的生产费用,也就是靠他在劳动期间所消费的、等于他从农业得到的最低限度工资的生活资料总额。既然农业劳动被看成唯一的生产劳动,那末,把农业劳动同工业劳动区别开来的剩余价值形式,即\textbf{地租},就被看成剩余价值的唯一形式。

因此,在重农学派那里不存在资本的\textbf{利润}——真正的利润,而地租本身只不过是这种利润的一个分枝。重农学派认为利润只是一种较高的工资,这种工资由土地所有者支付,并且由资本家作为收入来消费(因此,它完全象普通工人所得的最低限度的工资一样,加入生产费用),它增大原料的价值,因为它加入资本家即工业家在生产产品、变原料为新产品时的消费费用。

因此,某些重农主义者,例如老米拉波,把\textbf{货币利息}形式的剩余价值——利润的另一分枝——称为违反自然的高利贷。相反,杜尔哥认为货币利息是正当的,因为货币资本家本来可以购买土地,即购买地租,所以他的货币资本应当使他得到他把这笔资本变成地产时所能得到的那样多的剩余价值。由此可见,根据这种看法,连货币利息也不是新创造的价值,不是剩余价值;这里只是说明土地所有者得到的剩余价值的一部分为什么会以利息形式流到货币资本家手里,正如用别的理由[225]说明这个剩余价值的一部分为什么会以利润形式流到工业资本家手里一样。按照重农学派的意见,既然\textbf{农业劳动}是唯一的生产劳动,是唯一创造剩余价值的劳动,那末,把农业劳动同其他一切劳动部门区别开来的\textbf{剩余价值形式},即\textbf{地租},就是\textbf{剩余价值的一般形式}。工业利润和货币利息只是地租依以进行分配的各个不同项目,地租按照这些项目以一定的份额从土地所有者手里转到其他阶级手里。这同从亚当·斯密开始的后来的政治经济学家所持的观点完全相反,因为这些政治经济学家正确地把\textbf{工业利润}看成剩余价值\textbf{最初}为资本占有的\textbf{形式},从而看成剩余价值的最初的一般形式,而把利息和地租仅仅解释为由工业资本家分配给剩余价值共同占有者各阶级的工业利润的分枝。

除了上面所说的理由,即农业劳动是一种使剩余价值的创造在物质上显而易见,并且可以不经过流通过程就表现出来的劳动,重农学派还有一些别的理由说明他们的观点。

\textbf{第一},在农业中,地租表现为第三要素,表现为一种在工业中或者根本不存在,或者只是转瞬即逝的剩余价值形式。这是超过剩余价值(超过利润)的剩余价值,因此是最显而易见和最引人注目的剩余价值形式,是二次方的剩余价值。

\begin{quote}粗俗的政治经济学家\textbf{卡尔·阿伦德}(《合乎自然的国民经济学》1845 年哈瑙版第 461—462 页)说:“农业以地租形式创造一种在工业和商业中遇不到的价值:一种在补偿全部支付了的工资和全部消耗了的资本利润之后剩下来的价值。”\end{quote}

\textbf{第二},如果撇开对外贸易(重农学派为了抽象地考察资产阶级社会,完全正确地这样做了,而且应当这样做),那末很明显,从事加工工业等等而完全脱离农业的工人(斯图亚特称之为“自由人手”)的数目,取决于农业劳动者所生产的超过自己消费的农产品的数量。

\begin{quote}“显然,不从事农业劳动而能生活的人的相对数,完全取决于土地耕种者的劳动生产率。”(\textbf{理查·琼斯}《论财富的分配》1831 年伦敦版第 159—160 页)\end{quote}

可见,农业劳动不仅对于农业领域本身的剩余劳动来说是自然基础(关于这一点见前面的一个稿本)\endnote{马克思指他的 1861—1863 年手稿第 III 本第 105—106 页,在那里他也顺便提到了重农学派(《绝对剩余价值》一节,《剩余劳动的性质》一小节)。——第 22 页。},而且对于其他一切劳动部门之变为独立劳动部门,从而对于这些部门中创造的剩余价值来说,也是自然基础;因此很明显,只要价值实体被认为是一定的具体劳动,而不是抽象劳动及其尺度即劳动时间,农业劳动就必定被看作是剩余价值的创造者。

[226]\textbf{第三},一切剩余价值,不仅相对剩余价值,而且绝对剩余价值,都是以一定的劳动生产率为基础的。如果劳动生产率只达到这样的发展程度:一个人的劳动时间只够维持他本人的生活,只够生产和再生产他本人的生活资料,那就没有任何剩余劳动和任何剩余价值,就根本没有劳动能力的价值和这个劳动能力所创造的价值之间的差额了。因此,剩余劳动和剩余价值的可能性要以一定的劳动生产率为条件,这个生产率使劳动能力能够创造出超过本身价值的新价值,能够生产比维持生活过程所必需的更多的东西。而且,正象我们在\textbf{第二}点已经看到的,这个生产率,这个作为出发前提的生产率阶段,必定首先存在于农业劳动中,因而表现为\textbf{自然的赐予,自然的生产力}。在这里,在农业中,自然力的协助——通过运用和开发自动发生作用的自然力来提高人的劳动力,从一开始就具有广大的规模。在工业中,自然力的这种大规模的利用是随着大工业的发展才出现的。农业的一定发展阶段,不管是本国的还是外国的,是资本发展的基础。就这点来说,绝对剩余价值同相对剩余价值是一致的。(连重农学派的大敌\textbf{布坎南}都用这一点来反对亚·斯密,力图证明,甚至在现代城市工业产生之前,已先有农业的发展。)

\textbf{第四},因为重农学派的功绩和特征在于,它不是从流通中而是从生产中引出价值和剩余价值,所以它同货币主义和重商主义体系相反,必然从这样的生产部门开始,这个生产部门一般可以同流通、交换脱离开来单独考察,并且是不以人和人之间的交换为前提,而只以人和自然之间的交换为前提的。

\tsection{[(2)重农学派体系的矛盾:这个体系的封建主义外貌和它的资产阶级实质;对剩余价值的解释中的二重性]}

从上述情况也就产生了重农学派体系的矛盾。

实际上这是第一个对资本主义生产进行分析,并把资本在其中被生产出来又在其中进行生产的那些条件当作生产的永恒自然规律来表述的体系。但是另一方面,这个体系宁可说是封建制度即土地所有权统治的资产阶级式的再现;而资本最先得到独立发展的工业部门,在它看来却是“非生产的”劳动部门,只不过是农业的附庸而已。资本发展的第一个条件,是土地所有权同劳动分离,是土地——这个劳动的最初条件——作为独立的力量,作为掌握在特殊阶级手中的力量,开始同自由劳动者相对立。因此,在重农学派的解释中,土地所有者表现为真正的资本家,即剩余劳动的占有者。可见,在这里,封建主义是从资产阶级生产的角度来加以表述和说明的,而农业则被解释成唯一进行资本主义生产即剩余价值生产的生产部门。这样,封建主义就具有了资产阶级的性质,资产阶级社会获得了封建主义的外观。

这个外观曾迷惑了魁奈医生的贵族出身的门徒们,例如守旧的怪人老\textbf{米拉波}。在那些眼光比较远大的重农主义体系[227]代表者那里,特别是在\textbf{杜尔哥}那里,这个外观完全消失了,重农主义体系就成为在封建社会的框子里为自己开辟道路的新的资本主义社会的表现了。因而,这个体系是同刚从封建主义中孵化出来的资产阶级社会相适应的。所以出发点是在法国这个以农业为主的国家,而不是在英国这个以工业、商业和航海业为主的国家。在英国,目光自然集中到流通过程,看到的是产品只有作为一般社会劳动的表现,作为货币,才取得价值,变成商品。因此,只要问题涉及的不是价值形式,而是价值量和价值增殖,那末在这里首先看到的就是“\textbf{让渡利润}”,即斯图亚特所描述的相对利润。但是,如果要证明剩余价值是在生产领域本身创造的,那末,首先必须从剩余价值不依赖流通过程就能表现出来的劳动部门即农业着手。因而这方面的首创精神,是在一个以农业为主的国家中表现出来的。在重农学派的前辈老作家中,已经可以零星地看到近似重农学派的思想,例如在法国的布阿吉尔贝尔那里就可以部分地看到。但是这些思想只有在重农学派那里,才成为标志着科学新阶段的体系。

农业劳动者只能得到最低限度的工资,即“最必需品”,而他们再生产出来的东西却多于这个“最必需品”,这个余额就是地租,就是由劳动的基本条件——自然——的所有者占有的\textbf{剩余价值}。因此,重农学派不是说:劳动者是超过再生产他的劳动能力所必需的劳动时间进行劳动的,所以他创造的价值高于他的劳动能力的价值,换句话说,他付出的劳动大于他以工资形式得到的劳动量。但是他们说:劳动者在生产时消费的使用价值的总和小于他所生产的使用价值的总和,因而剩下一个使用价值的余额。——如果他只用再生产自己的劳动能力所必需的时间来进行劳动,那就没有什么余额了。但是重农学派只抓住这样一点:土地的生产力使劳动者能够在一个工作日(假定为已知量)生产出多于他维持生活所必需消费的东西。这样一来,这个剩余价值就表现为\textbf{自然的赐予},在自然的协助下,一定量的有机物(种子、畜群)使劳动能够把更多的无机物变为有机物。

另一方面,不言而喻,这里是假定土地所有者作为资本家同劳动者相对立的。土地所有者向劳动者支付劳动能力的代价,——这种劳动能力是劳动者当作商品提供给他的,——而作为补偿,他不但得到一个等价物,而且占有这种劳动能力所创造的价值增殖额。在这个交换中,必须以劳动的物质条件和劳动能力本身彼此脱离为前提。出发点是封建土地所有者,但他表现为一个资本家,表现为一个纯粹的商品所有者,他使他用来同劳动交换的商品的价值增殖,并且不仅收回这些商品的等价物,还收回超过这个等价物的余额,因为他把劳动能力只当作商品来支付代价。他作为商品所有者而同自由工人相对立。换句话说,这个土地所有者实质上是资本家。在这方面重农主义体系也是对的,因为劳动者同土地和土地所有权的分离[228]是资本主义生产和资本的生产的基本条件。

因此,在这一体系中就产生了以下矛盾:它最先试图用对于别人劳动的占有来解释\textbf{剩余价值},并且根据商品交换来解释这种占有,但是在它看来,价值不是社会劳动的形式,剩余价值不是剩余劳动;价值只是使用价值,只是物质,而剩余价值只是自然的赐予,——自然还给劳动的不是既定量的有机物,而是较大量的有机物。一方面,地租,即土地所有权的实际经济形式,脱去了土地所有权的封建外壳,归结为超出工资之上的纯粹的剩余价值。另一方面,这个剩余价值——又按封建主义的精神——是从自然而不是从社会,是从对土地的关系而不是从社会关系引伸出来的。价值本身只不过归结为使用价值,从而归结为物质。而在这个物质中,重农学派所关心的只是量的方面,即生产出来的使用价值超过消费掉的使用价值的余额,因而只是使用价值相互之间的量的关系,只是它们的最终要归结为劳动时间的交换价值。

这一切都是资本主义生产初期的矛盾,那时资本主义生产正从封建社会内部挣脱出来,暂时还只能给这个封建社会本身以资产阶级的解释,还没有找到它本身的形式;这正象哲学一样,哲学最初在意识的宗教形式中形成,从而一方面它消灭宗教本身,另一方面从它的积极内容说来,它自己还只在这个理想化的、化为思想的宗教领域内活动。

因此,在重农学派本身得出的结论中,对土地所有权的表面上的推崇,也就变成了对土地所有权的经济上的否定和对资本主义生产的肯定。一方面,全部赋税都转到地租上,换句话说,土地所有权部分地被没收了——而这正是法国革命制定的法律打算实施的办法,也是李嘉图学派的充分发展的现代政治经济学\endnote{马克思指激进的李嘉图学派。这个学派从李嘉图的理论中得出了反对土地私有制存在的实际结论,建议把这一制度(全部或部分)转变为资产阶级国家所有制。属于这个激进的李嘉图学派的有詹姆斯·穆勒、约翰·斯图亚特·穆勒、希尔迪奇,在一定程度上也有舍尔比利埃。关于这一点见本卷第二册,马克思手稿第 458 页;第三册,马克思手稿第 791、1120 和 1139 页;并见《哲学的贫困》(《马克思恩格斯全集》中文版第 4 卷第 187 页)和马克思 1881 年 6 月 20 日给左尔格的信(见《马克思恩格斯全集》中文版第 35 卷第 190—194 页)。——第 26、42 页。}的最终结论。因为地租被认为是唯一的剩余价值,并且根据这一点,一切赋税都落到地租身上,所以对其他形式的收入课税,只不过是对土地所有权采取间接的、因而在经济上有害的、妨碍生产的课税办法。结果,赋税的负担,从而国家的各种干涉,都落不到工业身上,工业也就摆脱了国家的任何干涉。这样做,表面上是有利于土地所有权,不是为了工业的利益而是为了土地所有权的利益。

与此有关的是:自由放任\endnote{自由放任(原文是:laissezfaire,laissezaller,亦译听之任之)是重农学派的口号。重农学派认为,经济生活是受自然规律调节的,国家不得对经济事务进行干涉和监督;国家用各种规章进行干涉,不仅无益,而且有害;他们要求实行自由主义的经济政策。——第 27、42、162 页。},无拘无束的自由竞争,工业摆脱国家的任何干涉,取消垄断等等。按照重农学派的意见,既然工业什么也不创造,只是把农业提供给它的价值变成另一种形式;既然它没有在这个价值上增加任何新价值,只是把提供给它的价值以另一种形式作为等价物归还,那末,很自然,最好是这个转变过程不受干扰地、最便宜地进行,而要达到这一点,只有通过自由竞争,听任资本主义生产自行其是。这样一来,把资产阶级社会从建立在封建社会废墟上的君主专制下解放出来,就只是为了[229]已经变成资本家并一心一意想发财致富的封建土地所有者的利益。资本家成为仅仅为了土地所有者的利益的资本家,正象进一步发展了的政治经济学让资本家成为仅仅为了工人阶级的利益的资本家一样。

从上述一切可以看到,现代的经济学家,如出版了重农学派的著作和自己论述重农学派的得奖论文的欧仁·德尔先生,认为重农学派关于只有农业劳动才具有生产性、关于地租是剩余价值的唯一形式、关于土地所有者在生产体系中占杰出地位这些独特的论点,同重农学派的自由竞争的宣传、大工业和资本主义生产的原则毫无联系,只是偶然地凑合在一起,——他们这种看法是多么不了解重农学派。同时也就可以理解,这个体系的封建主义外观——完全象启蒙时代的贵族腔调——必然会使不少的封建老爷成为这个实质上是宣告在封建废墟上建立资产阶级生产制度的体系的狂热的拥护者和传播者。

\tsection{[(3)魁奈论社会的三个阶级。杜尔哥对重农主义理论的进一步发展:对资本主义关系作更深入分析的因素]}

现在我们来考察几段引文,一方面为了阐明上述论点,一方面为了给以证明。

在\textbf{魁奈}的《经济表的分析》一书中,国民由三个市民阶级组成:

\begin{quote}“\textbf{生产阶级}〈农业劳动者〉、\textbf{土地所有者阶级}”和“\textbf{不生产阶级}〈“所有从事农业以外的其他职务和其他工作的市民”〉”。\authornote{本卷引文中凡是尖括号〈〉和花括号\fontbox{~\{}\fontbox{\}~}内的话都是马克思加的。——译者注}(《重农学派》,欧仁·德尔出版,1846 年巴黎版第 1 部第 58 页)\end{quote}

只有农业劳动者才是生产阶级,创造剩余价值的阶级,土地所有者就不是。土地所有者阶级不是“不生产的”,因为它代表“剩余价值”,这个阶级的重要并不是由于它创造这个剩余价值,而完全是由于它占有这个剩余价值。

在\textbf{杜尔哥}那里,重农主义体系发展到最高峰。他的著作中某些地方甚至把“纯粹的自然赐予”看作\textbf{剩余劳动},另一方面,他用劳动者脱离劳动条件、劳动条件作为拿这些条件做买卖的那个阶级的财产同劳动者相对立这种情况,来说明工人提供的东西必须超过维持生活的工资。

说明只有农业劳动是生产劳动的第一个理由是:农业劳动是其他一切劳动得以独立存在的自然基础和前提。

\begin{quote}“他的〈土地耕种者的〉劳动,在社会不同成员所分担的各种劳动中占着首要地位……正象在社会分工以前,人为取得食物而必须进行的劳动,在他为满足自己的各种需要而不得不进行的各种劳动中占着首要地位一样。这不是在荣誉或尊严的意义上的首要地位;这是由\textbf{生理的必然性}决定的首要地位……土地耕种者的劳动使土地生产出超过他本人需要的东西,这些东西是社会其他一切成员用自己的劳动换来的工资的唯一基金。当后者利用从这种交换中得来的报酬再来购买土地耕种者的产品时,他们归还土地耕种者的〈在物质形式上〉,恰好只是他们原来得到的。这就是这两种劳动之间的本质[230]差别。”(《关于财富的形成和分配的考察》(1766 年),载于德尔出版的《杜尔哥全集》1844 年巴黎版第 1 卷第 9—10 页)\end{quote}

剩余价值究竟是怎样产生的呢\fontbox{?}它不是从流通中产生的,但是它在流通中实现。产品是按自己的价值出卖的,不是\textbf{高于}自己的价值出卖的。这里没有价格超过价值的余额。但是,因为产品按它的价值出卖,卖者就实现了剩余价值。这种情况所以可能,只是因为卖者本人对他所卖的价值没有全部支付过代价,换句话说,因为产品中包含着卖者没有支付过代价的、没有用等价物补偿的价值组成部分。农业劳动的情况正是这样。卖者出卖他没有买过的东西。杜尔哥最初把这个没有买过的东西描绘成“\textbf{纯粹的自然赐予}”。但是我们将会看到,这个“纯粹的自然赐予”在他那里,不知不觉地变成土地所有者没有买过而以农产品形式出卖的土地耕种者的剩余劳动。

\begin{quote}“土地耕种者的劳动一旦\textbf{生产出超过}他的需要\textbf{的东西},他就可以用这个余额——\textbf{自然给他的}超过他的劳动报酬的\textbf{纯粹的赐予}——去购买社会其他成员的劳动。后者向他出卖自己的劳动时所得到的只能维持生活;而土地耕种者除了自己的生存资料以外,还得到一种独立的和可以自由支配的财富,这是\textbf{他没有买却拿去卖的}财富。因此,他是财富(财富通过自己的流通使社会上一切劳动活跃起来)的唯一源泉,\textbf{因为他的劳动是唯一生产出超过劳动报酬的东西的劳动}。”(同上,第 11 页)\end{quote}

在这第一个解释中,第一,掌握了剩余价值的本质,就是说,剩余价值是卖者没有支付过等价物,即没有买过而拿去出卖时实现的价值。它是\textbf{没有支付过代价的价值}。但是第二,这个超过“\textbf{劳动报酬}”的余额被看成是“纯粹的自然赐予”,因为劳动者在他的工作日中所能生产的东西,比再生产他的劳动能力所必需的东西多,比他的工资多,这种情况一般地说就是自然的赐予,是取决于自然的生产率的。按照这第一个解释,全部产品还是归劳动者本人占有。但它分成两部分。第一部分形成劳动者的工资——他被看作是自己的雇佣劳动者,他把再生产他的劳动能力,维持他的生活所必需的那部分产品支付给自己。除此以外的第二部分是\textbf{自然的赐予},形成剩余价值。但是,只要抛开“土地耕种者-土地所有者”这个前提,而产品的两部分即工资和剩余价值分别属于不同的阶级,一部分属于雇佣劳动者,另一部分属于土地所有者,那末,这个剩余价值的性质,这个“纯粹的自然赐予”的性质,就更清楚地表现出来了。

不论工业还是农业本身,要形成雇佣劳动者阶级(最初,一切从事工业的人只表现为“土地耕种者-土地所有者”的雇工,即雇佣劳动者),劳动条件必须同劳动能力分离,而这种分离的基础是,土地本身表现为社会上一部分人的私有财产,以致社会上另一部分人失去了借以运用自己劳动的这个物质条件。

\begin{quote}“在最初的时代,土地所有者同土地耕种者还没有区别……在那个最初的时代,每一个勤劳的人要多少土地,就可以找到多少土地,[231]谁也不会想到\textbf{为别人劳动}……但是,到了最后,每一块土地都有了主人,那些没有得到土地的人最初没有别的出路,只好从事\textbf{雇佣}阶级的职业〈即手工业者阶级,一句话,一切非农业劳动者阶级〉,\textbf{用自己双手的劳动去换取}土地耕种者-土地所有者的产品余额。”(第 12 页)“土地耕种者-土地所有者”可以用土地对其劳动所赐予的“相当多的余额支付别人,要别人为他耕种土地。对于靠工资过活的人来说,无论从事哪种劳动来挣工资,都是一样。\textbf{因此,土地所有权必定要同农业劳动分离,而且不久也真的分离了}……土地所有者开始把耕种土地的劳动交给雇佣的土地耕种者去担负”。(同上,第 13 页)\end{quote}

这样,资本和雇佣劳动的关系就在农业中出现了。只有当一定数量的人丧失对劳动条件——首先是土地——的所有权,并且除了自己的劳动之外再也没有什么可以出卖的时候,这种关系才会出现。

现在,对于已经不能生产任何商品而不得不出卖自己的劳动的雇佣工人来说,\textbf{最低限度}的工资,即必要生活资料的等价物,必然成为他同劳动条件的所有者交换时的规律。

\begin{quote}“只凭双手和勤劳的普通工人,除了能够把他的劳动出卖给别人以外,就一无所有……在一切劳动部门,工人的工资都必定是,而实际上也是限于维持他的生活所必需的东西。”(同上,第 10 页)\end{quote}

而且,雇佣劳动一出现,

\begin{quote}“土地产品就分成两部分:一部分包括土地耕种者的生存资料和利润,这是他的劳动的报酬,也是他耕种土地所有者的土地的条件;余下的就是那个独立的可以自由支配的部分,这是\textbf{土地作为纯粹的赐予交给耕种土地的人的}、超过他的预付和他的劳动报酬的部分;这是归土地所有者的份额,或者说,是土地所有者赖以不劳动而生活并且可以任意花费的收入”。(同上,第 14 页)\end{quote}

但是,这个“纯粹的土地赐予”现在已经明确地表现为土地给“耕种土地的人”的礼物,即土地给劳动的礼物,表现为用在土地上的劳动的生产力,这种生产力是劳动由于利用自然的生产力所具有的,从而是劳动从土地中吸取的,是劳动只作为劳动从土地中吸取的。因此,在土地所有者手中,余额已经不再表现为“自然的赐予”,而表现为对于别人劳动的——不给等价物的——占有,后者的劳动由于自然的生产率,能够生产出超过本身需要的生存资料,但是它由于是雇佣劳动,在全部劳动产品中只能占有“维持他的生活所必需的东西”。

\begin{quote}“\textbf{土地耕种者}生产\textbf{他自己的工资},此外还生产用来支付整个手工业者和其他雇佣人员阶级的收入。\textbf{土地所有者没有土地耕种者的劳动,就一无所有}〈可见不是靠“纯粹的自然赐予”〉;他从土地耕种者那里[232]得到他的生存资料和用来支付其他雇佣人员劳动的东西……土地耕种者需要土地所有者,却仅仅由于习俗和法律。”(同上,第 15 页)\end{quote}

可见,在这里,剩余价值直接被描绘成土地所有者不给等价物而占有的土地耕种者劳动的一部分,因而这部分劳动的产品是他没有买过而拿去出卖的。但是,杜尔哥所指的不是交换价值本身,不是劳动时间本身,而是土地耕种者的劳动超出他自己的工资之上提供给土地所有者的产品余额;但这个产品余额,只不过是土地耕种者在他为再生产自己的工资而劳动的时间以外,白白地为土地所有者劳动的那一定量时间的体现。

因此,我们看到,重农学派在\textbf{农业劳动范围内}是正确地理解剩余价值的,他们把剩余价值看成雇佣劳动者的劳动产品,虽然对于这种劳动本身,他们又是从它表现为使用价值的具体形式来考察的。

顺便指出,杜尔哥认为,农业的资本主义经营方式——“土地出租”是

\begin{quote}“一切方式中最有利的方式,但是采用这种方式应以已经富庶的地区为前提”。(同上,第 21 页)\end{quote}

\fontbox{~\{}在考察剩余价值时,必须从流通领域转到生产领域,就是说,不是简单地从商品同商品的交换中,而是从劳动条件的所有者和工人之间在生产范围内进行的交换中,引出剩余价值。而劳动条件的所有者和工人又是作为商品所有者彼此对立的,因此,这里决不是以脱离交换的生产为前提。\fontbox{\}~}

\fontbox{~\{}在重农主义体系中,土地所有者是“雇主”,而其他一切生产部门的工人和企业主是“工资所得者”,或者说,“雇佣人员”。由此也就有了“管理者”和“被管理者”。\fontbox{\}~}

杜尔哥这样分析劳动条件:

\begin{quote}“在任何劳动部门,劳动者事先都要有劳动工具,都要有足够数量的材料作为他的劳动对象;而且都要在他的成品出卖之前有可能维持生活。”(同上,第 34 页)\end{quote}

所有这些“预付”,这些使劳动有可能进行,因而成为劳动过程的\textbf{前提}的条件,最初是由土地无偿提供的:

\begin{quote}“在土地完全没有耕种以前,土地就提供了最初的预付基金”,如果实、鱼、兽类等等,还有工具——例如树枝、石块、家畜,后者的数量由于繁殖而增加起来,它们每年还提供“乳、毛、皮和其他材料,这些产品连同从森林里采伐来的木材一起,成了工业生产的最初基金”。(同上,第 34 页)\end{quote}

这些劳动条件,这些“预付”,一旦必须由第三者预付给工人,就变成\textbf{资本},而这种情况,从工人除了本身的劳动能力外一无所有的时候起,就出现了。

\begin{quote}“\textbf{当}社会上大部分成员\textbf{只靠自己的双手谋生的时候},这些靠工资生活的人,无论是为了取得加工的原料,还是为了在发工资之前维持生活,都必须\textbf{事先}得到\textbf{一些东西}。”(同上,第 37—38 页)\end{quote}

[233]杜尔哥给“\textbf{资本}”下的定义是

\begin{quote}“积累起来的流动的价值”。(同上,第 38 页)最初,土地所有者或土地耕种者每天直接把工资和材料付给,比如说,纺麻女工。随着工业的发展,必须使用较大量的“预付”,并保证这个生产过程的不断进行。于是这件事就由“资本所有者”担当起来。这些“资本所有者”必须在自己产品的价格中收回他的全部“\textbf{预付}”,取得等于“假定他用货币购买一块〈土地〉而给他带来的东西”的一笔利润,还要取得他的“工资”,“因为,毫无疑问,如果利润一样多,他就宁可毫不费力地靠那笔资本能够买到的土地的收入来生活了”。(第 38—39 页)\end{quote}

“工业雇佣阶级”又划分为

\begin{quote}“企业资本家和普通工人”等等。(第 39 页)\end{quote}

“租地农场企业主”的情形也和这些企业资本家的情形一样。他们也象上述情况一样,必须收回全部“预付”,同时取得利润。

\begin{quote}“所有这一切都必须从土地产品的价格中事先扣除;\textbf{余下的部分}由土地耕种者交给土地所有者,作为允许他利用后者的土地来建立企业的报酬。这就是租金,就是土地所有者的收入,就是\textbf{纯产品},因为土地生产出来补偿各种预付和这些预付的提供者的利润的全部东西,不能看成收入,而只能看成\textbf{土地耕作费用的补偿};要知道,如果土地耕种者收不回这些费用,他就不会花费自己的资金和劳动去耕种别人的土地。”(同上,第 40 页)\end{quote}

最后:

\begin{quote}“虽然资本有一部分是由劳动者阶级的利润积蓄而成,但是,既然这些利润总是来自土地(因为所有这些利润不是由收入来支付,便是由生产这种收入的费用来支付),那末很明显,资本也完全象收入一样,来自土地;或者更确切地说,资本不外是土地所生产的一部分价值的积累,这一部分价值是收入的所有者或分享者可以每年储存起来,而不用来满足自己的需要的。”(第 66 页)\end{quote}

不言而喻,既然地租成为剩余价值的唯一形式,那末唯有地租才是资本积累的源泉。资本家在此以外积累的东西,是从他们的“工资”中(从供他们消费的收入中,因为利润正是被看成这种收入)积攒下来的。

因为利润和工资一样,算在土地耕作费用中,只有余下的部分才成为土地所有者的收入,所以土地所有者尽管被摆在可敬的地位,事实上,丝毫不分摊土地耕作费用,因而他不再是生产当事人——这一点同李嘉图学派的看法完全一样。

重农主义的产生,既同反对柯尔培尔主义\endnote{指法国路易十四时期柯尔培尔的重商主义经济政策。柯尔培尔当时任财政总稽核,他采取的财政经济政策是为了巩固专制国家的。例如,改革税收制度,建立垄断性的对外贸易公司,通过统一关税率来促进国内贸易,建立国家工场手工业,以及修建道路和港口。柯尔培尔主义客观上促进了新兴的资本主义经济方式的发展。它是法国资本原始积累的一个工具。但是随着资本主义生产方式日益强大,国家的这些强制性措施就由于日益阻碍经济发展而失去作用。——第 35、42 页。}有关系,又特别是同罗氏制度的破产\endnote{英国银行家和经济学家约翰·罗于 1716 年在巴黎创办一家私人银行,该银行于 1718 年改组为国家银行。罗力图依靠这家银行来实现他的荒唐主张,即国家通过发行不可兑的银行券来增加国内财富。罗氏银行无限制地发行纸币,同时回收金属货币。结果,交易所买空卖空和投机倒把活动空前盛行。到 1720 年,国家银行倒闭,罗氏“制度”也就彻底破产。罗逃往国外。——第 35、40 页。}有关系。

\tsection{[(4)把价值同自然物质混淆起来(帕奥累蒂)]}

[234]把价值同自然物质混淆起来,或者确切些说,把两者等同起来的看法,以及这种看法同重农学派的整套见解的联系,在后面这段引文中表现得很清楚。这段引文摘自\textbf{斐迪南多·帕奥累蒂}的著作《谋求幸福社会的真正手段》(这部著作一部分是针对维里的,维里在他的《政治经济学研究》(1771 年)中曾经反对重农学派)。(托斯卡纳的帕奥累蒂所写的这部著作,见库斯托第出版的《意大利政治经济学名家文集》(现代部分)第 20 卷。)

\begin{quote}象“土地产品”这样的“\textbf{物质数量倍增的情况}”,“在工业中无疑是没有的,而且永远也不可能发生,因为工业只给物质以形式,仅仅使物质发生形态变化;所以工业什么也不创造。但是,有人反驳我说,工业既给物质以形式,那它就是生产的;因为它即使不是物质的生产,也还是形式的生产。好吧,我不否认这一点;可是,\textbf{这不是财富的创造,相反,这无非是一种支出}……作为政治经济学的前提和研究对象的,是物质的和实在的生产,而这种生产只能在农业中发生,因为只有农业才使构成财富的物质和产品的数量倍增……工业从农业购买原料,以便把它加工。工业劳动,前面已经说过,只给这个原料以形式,但什么也不给它添加,不能使它倍增”。(第 196—197 页)“给厨师一定数量的豌豆,要他用来准备午餐;他好好烹调之后,将烧好的豌豆端到你桌上,但是数量同他拿去的一样;相反,把同量的豌豆交给种菜人,让他把豌豆拜托给土地,到时候,他归还给你的至少比他领去的多 3 倍。这才是真正的唯一的生产。”(第 197 页)“物由于人的需要才有价值。因此,商品的价值,或这个价值的增加,不是工业劳动的结果,而是劳动者支出的结果。”(第 198 页)“一种最新的工业品刚一出现,它很快就在国内外风行起来;可是,其他工业家、商人的竞争会\textbf{极快地}把它的价格压低到它应有的水平,这个水平……决定于原料和工人生存资料的价值。”(第 204—205 页)\end{quote}


% --------------------------------------

\cleardoublepage

\pagestyle{endnotestyle}

\printendnotes

\addcontentsline{toc}{chapter}{注释}

% --------------------------------------

\end{document}

% ------------------------------------------------------------------------------
% EOF
% ------------------------------------------------------------------------------
